\section{Other Subclasses of Stochastic Games}

SPGs were motivated by the desire to make all continuation games be potential games. This required the stochastic game to have modular dynamics. In contrast, stochastic zero sum games automatically have this feature, without any requirements on their transition probabilities. We may use this general template the construct new classes of stochastic games:

\begin{enumerate}
    \item Start with an interesting class $\mathcal{C}$ of normal form games. 
    \item Identify constraints on a stochastic games such that all of its continuation games belong to $\mathcal{C}$.
    \item Use these constraints to define a subclass of stochastic games.
\end{enumerate}

One class of games that could be studied in this way are supermodular games. These games are interesting from a learning perspective as there are learning methods that converge to Nash equilibrium \cite{hofbauer2002global}. Furthermore, supermodular games have interesting equilibria sets. They also have a number of economic applications \cite{topkis1979equilibrium, vives1990nash, milgrom1990rationalizability}.